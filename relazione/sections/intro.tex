\section{Introduzione}\label{sec:introduction}
L'ottimizzazione del portafoglio (PO) è un'attività finanziaria di primaria 
importanza, con applicazioni significative in diversi contesti, come 
i fondi di investimento, i piani pensionistici e altre strategie di 
allocazione del capitale. Data una disponibilità di budget e/o un 
insieme di asset, l'obiettivo è individuare operazioni ottimali 
all'interno di un mercato che può includere un numero elevato di asset.
La corretta allocazione degli asset ha un impatto diretto sulla 
redditività degli investimenti, consentendo di ottenere rendimenti più 
elevati e una migliore gestione del rischio. Data la rilevanza economica 
del problema, l'ottimizzazione del portafoglio rappresenta un'area 
strategica sia per le istituzioni finanziarie sia per gli investitori.

\paragraph{Limiti dei metodi classici}
I metodi tradizionali, come gli approcci geometrici o gli algoritmi 
euristici, presentano significative limitazioni, soprattutto in termini 
di scalabilità ed efficienza. Con l'aumentare della complessità e delle 
dimensioni del mercato, la risoluzione del problema diventa rapidamente 
intrattabile per i computer classici. Ad esempio, algoritmi come il 
branch-and-bound~\cite{land2010automatic}, utilizzati per trovare soluzioni 
esatte, faticano a gestire mercati con un numero elevato di asset. 

\paragraph{Quantum computing}
L'introduzione del calcolo quantistico apre nuove possibilità per 
affrontare i limiti dei metodi classici. Sfruttando i principi della 
meccanica quantistica, come la sovrapposizione e l'entanglement, i 
computer quantistici promettono di risolvere problemi di ottimizzazione 
in modo più efficiente. In particolare, i problemi di ottimizzazione 
quadratica, come quello del portafoglio, possono beneficiare di 
algoritmi quantistici in grado di trovare soluzioni quasi ottimali in 
tempi significativamente ridotti rispetto ai metodi tradizionali.

\paragraph{Computer classici vs computer quantistici}
I computer classici (CPU) elaborano le informazioni utilizzando i bit, che 
possono assumere esclusivamente due stati, 0 o 1. Questa caratteristica 
limita la capacità di esplorare lo spazio delle soluzioni in parallelo, 
costringendo i calcoli a procedere in modo sequenziale o attraverso 
tecniche di parallelismo limitate.

Al contrario, i computer quantistici (QPU) sfruttano i qubit, che possono 
trovarsi in una sovrapposizione di stati, rappresentando simultaneamente 
sia 0 che 1. Grazie a questa proprietà unica, i computer quantistici 
sono in grado di eseguire calcoli in parallelo, esplorando uno spazio 
di soluzioni molto più vasto rispetto ai computer classici e rendendoli 
particolarmente adatti per affrontare problemi complessi come quelli di 
ottimizzazione.
