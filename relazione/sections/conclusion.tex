\section{Conclusioni}\label{sec:conclusion}

I risultati ottenuti mostrano chiaramente l'impatto significativo del rumore sulle prestazioni 
degli algoritmi di ottimizzazione del portafoglio. L'algoritmo VQE, sebbene capace di esplorare 
una maggiore varietà di soluzioni, dimostra una sensibilità elevata al rumore, che porta a una 
qualità complessiva inferiore delle soluzioni rispetto al QAOA. D'altra parte, l'algoritmo QAOA 
riesce a mantenere una buona qualità delle soluzioni e una maggiore robustezza in presenza di rumore, 
pur mostrando una capacità esplorativa limitata.

Questi risultati evidenziano l'importanza cruciale di sviluppare tecniche di mitigazione del rumore 
per migliorare le prestazioni degli algoritmi quantistici in ambienti reali. Tuttavia, è importante 
sottolineare che, con la tecnologia attuale, l'utilizzo del quantum computing per risolvere problemi 
di ottimizzazione del portafoglio su larga scala, che richiederebbero migliaia di qubit, risulta 
impraticabile. La quantità di rumore introdotta in tali scenari sarebbe troppo elevata, compromettendo 
la qualità delle soluzioni proposte e rendendo inefficace l'approccio quantistico.