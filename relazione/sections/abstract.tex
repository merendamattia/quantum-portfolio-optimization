\begin{abstract}
    In questo elaborato esploriamo l'applicazione del calcolo quantistico
    all'ottimizzazione del portafoglio in ambito finanziario, confrontando i metodi classici 
    con gli approcci quantistici basati su VQE (Variational Quantum Eigensolver) 
    e QAOA (Quantum Approximate Optimization Algorithm). 
    Formuliamo il problema dell'ottimizzazione del portafoglio come un 
    problema QUBO (Quadratic Unconstrained Binary Optimization) e lo 
    implementiamo utilizzando il framework Qiskit. Lo studio include 
    simulazioni sia in assenza che in presenza di rumore per valutare 
    le prestazioni degli algoritmi in condizioni realistiche. 
    La ricerca evidenzia le attuali limitazioni nel scalare l'ottimizzazione 
    quantistica del portafoglio alle applicazioni del mondo reale, 
    principalmente a causa dei vincoli hardware e dell'impatto del 
    rumore su sistemi più grandi. 
\end{abstract}